\documentclass[12pt, a4paper]{ctexart}
\usepackage{amsmath}
\usepackage{amssymb}
\usepackage{graphicx}
\usepackage{geometry}
\usepackage{booktabs} % 更好的表格线
\usepackage{url}

\geometry{
	a4paper,
	total={170mm,257mm},
	left=20mm,
	top=20mm,
}

\title{\textbf{SimpleForest 3D 场景渲染项目技术报告}}
\date{\today}

\begin{document}
	
	\maketitle
	\tableofcontents
	\newpage
	
	\section{项目概览与目标 (Executive Summary)}
	
	\begin{itemize}
		\item \textbf{项目名:} SimpleForest (简单森林)
		\item \textbf{功能:} 基于 OpenGL 的小场景渲染演示,包括程序化小屋、随机树木、天空盒、ImGui 实时参数面板及 Blinn-Phong 简单光照。
		\item \textbf{运行时特性:} 程序化几何作为 Assimp 模型的稳健后备;运行时纹理与材质开关控制。
		\item \textbf{核心目标:} 建立一个高模块化、易扩展的 3D 渲染框架,演示程序化建模、外部模型集成和实时参数调试的能力。
	\end{itemize}
	
	\section{高层架构与模块化设计 (High-Level Architecture)}
	本项目采用了严格的分层架构,所有功能代码按职责划分至相应的模块目录,确保了高度的解耦和可维护性。
	
	\subsection{项目目录结构}
	\begin{verbatim}
		/SimpleForest
		|-- include/ (第三方库)
		|-- src/
		|   |-- core/
		|   |-- geometry/
		|   |-- scene/
		|   |-- input/
		|   |-- main.cpp (程序入口)
		|-- shaders/ (着色器文件)
		|-- objects/ (纹理/模型文件)
	\end{verbatim}
	
	\subsection{模块职责划分}
	\begin{table}[h]
		\centering
		\caption{核心模块职责一览}
		\label{tab:architecture}
		\begin{tabular}{p{2cm} p{3.7cm} p{8.3cm}}
			\toprule
			\textbf{模块目录} & \textbf{核心文件} & \textbf{职责详细说明} \\
			\midrule
			\texttt{main.cpp} & N/A & 应用程序入口。负责初始化、资源加载、主循环框架和 ImGui 调度。 \\
			\midrule
			\texttt{core/} & \texttt{Shader.*} & 着色器封装 (load/use/setUniform)。 \\
			& \texttt{Camera.*} & 摄像机逻辑(视图矩阵计算、欧拉角管理)。 \\
			& \texttt{Texture.*} & 纹理加载(2D/Cubemap,基于 \texttt{stb\_image})。 \\
			& \texttt{Model.*} & Assimp 模型加载(归一化、VAO/VBO、Draw())。 \\
			& \texttt{PathUtils.*} & 资源路径处理与可移植性支持。 \\
			\midrule
			\texttt{geometry/} & \texttt{Mesh.h} & 网格数据结构定义(VAO/VBO/EBO + \texttt{indexCount})。 \\
			& \texttt{PrimitiveFactory.*} & 程序化网格生成(\texttt{createCube}, \texttt{createCone}, \texttt{createSkybox} 等)。 \\
			\midrule
			\texttt{scene/} & \texttt{Materials.h} & 全局材质参数和颜色定义(用于 Blinn-Phong)。 \\
			& \texttt{Tree.h} & \texttt{Tree} 结构体与随机树木生成 (\texttt{generateRandomTrees})。 \\
			& \texttt{HouseRenderer.h} & 小屋渲染逻辑封装 (\texttt{renderDetailedHouse})。 \\
			\midrule
			\texttt{input/} & \texttt{Input.*} & 集中处理 GLFW 键鼠回调,转化为场景控制。 \\
			\bottomrule
		\end{tabular}
	\end{table}
	
	\section{核心技术与实现细节 (Implementation Details)}
	
	\subsection{渲染管线流程 (Main Loop)}
	主循环遵循严格的顺序,以确保渲染正确性,特别是天空盒的特殊处理:
	
	\begin{enumerate}
		\item \textbf{预处理:} 计算 \texttt{deltaTime},处理用户输入 (\texttt{processInput})。
		\item \textbf{清屏:} \texttt{glClear} 并开启 \texttt{GL\_DEPTH\_TEST}。
		\item \textbf{场景渲染:} 激活 \texttt{basicShader},设置 Blinn-Phong 光照参数、视图 (\texttt{camera.getView()}) 和投影矩阵。
		\item \textbf{绘制房屋:} 调用 \texttt{renderDetailedHouse},绑定墙壁、屋顶等纹理。
		\item \textbf{绘制树木:} 执行模型加载/程序化回退逻辑(详见下节)。
		\item \textbf{绘制天空盒:}
		\begin{itemize}
			\item 特殊状态:\texttt{glDepthMask(GL\_FALSE)} 和 \texttt{glDepthFunc(GL\_LEQUAL)}。
			\item 激活 \texttt{skyboxShader},移除视图矩阵的平移分量 (\texttt{glm::mat4(glm::mat3(view))})。
			\item 绘制天空盒立方体并绑定 Cubemap 纹理。
		\end{itemize}
		\item \textbf{状态恢复:} 恢复 \texttt{glDepthMask(GL\_TRUE)} 和 \texttt{glDepthFunc(GL\_LESS)}。
		\item \textbf{ImGui 绘制:} 渲染 ImGui 界面。
		\item \textbf{交换缓冲:} \texttt{glfwSwapBuffers}。
	\end{enumerate}
	
	\subsection{模型的稳健加载与程序化回退机制}
	
	项目实现了 Assimp 外部模型加载与程序化几何体的稳健切换机制,确保了程序的鲁棒性。
	
	\begin{table}[h]
		\centering
		\caption{模型加载与回退机制}
		\label{tab:fallback}
		\begin{tabular}{p{2cm} p{3.5cm} p{7.5cm}}
			\toprule
			\textbf{阶段} & \textbf{机制} & \textbf{逻辑描述} \\
			\midrule
			编译时 & \texttt{ASSIMP\_AVAILABLE} & 宏控制 \texttt{Model.*} 模块的条件编译。 \\
			\midrule
			初始化 & 文件检查与加载 & \texttt{main.cpp} 尝试加载 \texttt{objects/tree.obj}。\\
			\midrule
			初始化 & 稳健回退 (Fallback) & 若加载失败 (\texttt{tree.obj} 不存在或 Assimp 错误),则 \texttt{treeModel} 被置为 \texttt{nullptr}。 \\
			\midrule
			渲染时 & 动态分支 & \texttt{if (treeModel != nullptr)} 则调用 \texttt{Model::Draw()};\textbf{否则} 使用 \texttt{cylinder} (树干) 和 \texttt{cone} (树冠) 程序化几何体进行绘制。 \\
			\bottomrule
		\end{tabular}
	\end{table}
	
	\subsection{交互式调试与控制 (ImGui)}
	\textbf{ImGui 实时参数面板} 提供了关键的运行时调试功能:
	\begin{itemize}
		\item \textbf{全局纹理开关 (\texttt{useTextureGlobally})}:快速在纹理贴图和纯色光照模式之间切换,用于分离调试纹理和光照计算。
		\item \textbf{材质/颜色调整}:实时修改小屋和树木的漫反射颜色 (\texttt{ColorEdit3})。
		\item \textbf{光照控制}:实时调整光源方向 (\texttt{lightDir}) 和颜色 (\texttt{lightColor})。
		\item \textbf{输入切换}:捕获/释放鼠标按钮,方便在 3D 视图和 UI 交互之间快速切换。
	\end{itemize}
	
\end{document}